\section{Data Analytics}
\label{chapter1_section2}

Within the IT department of CFM, the Data division is in charge of collecting, processing, and extracting information from the data. The data considered at CFM in general is quite diverse: market data (real time prices and trading orders, representing more than 10000 products and several To of data a day); macroeconomic indicators (national and regional statistics, closely monitoring the calendar of data release); corporate fundamentals (operating accounts); and alternative data (texts, graphs, forecasts...).

Within the Data division, The Data Analytics team consists of data engineers and data scientists working on the data pipelines that eventually lead to investment decisions. The Data Analytics team works mainly with two types of data. The first type is time-series, mainly for prediction purposes. Traditional machine learning models are used, along with tools like Shap or Eli5 for model understanding and Dash for data visualisation. The second type is alternative data like text or graphs, for which specialized libraries are used (TensorFlow or Torch for NLP, Neo4J or NetworkX for networks).

The projects in the team follow a flow from exploration to production, along 4 main steps: data engineering (access to providers and creation of data flow); data characterization (data history and detection of bias); value extraction (prediction with machine learning models); production and support (to create programmes that are well-documented, tested and sustainable).

Macro-financial datasets and time series enter the scope of the Data Analytics Division. In this respect, the Data Analytics team proposed in 2020 an internship more specifically oriented towards predicting (or rather, ``nowcasting'', see below) economic activity. This project constitutes the object or the present report.

